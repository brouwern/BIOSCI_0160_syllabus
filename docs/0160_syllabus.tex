% Options for packages loaded elsewhere
\PassOptionsToPackage{unicode}{hyperref}
\PassOptionsToPackage{hyphens}{url}
%
\documentclass[
]{book}
\usepackage{amsmath,amssymb}
\usepackage{lmodern}
\usepackage{iftex}
\ifPDFTeX
  \usepackage[T1]{fontenc}
  \usepackage[utf8]{inputenc}
  \usepackage{textcomp} % provide euro and other symbols
\else % if luatex or xetex
  \usepackage{unicode-math}
  \defaultfontfeatures{Scale=MatchLowercase}
  \defaultfontfeatures[\rmfamily]{Ligatures=TeX,Scale=1}
\fi
% Use upquote if available, for straight quotes in verbatim environments
\IfFileExists{upquote.sty}{\usepackage{upquote}}{}
\IfFileExists{microtype.sty}{% use microtype if available
  \usepackage[]{microtype}
  \UseMicrotypeSet[protrusion]{basicmath} % disable protrusion for tt fonts
}{}
\makeatletter
\@ifundefined{KOMAClassName}{% if non-KOMA class
  \IfFileExists{parskip.sty}{%
    \usepackage{parskip}
  }{% else
    \setlength{\parindent}{0pt}
    \setlength{\parskip}{6pt plus 2pt minus 1pt}}
}{% if KOMA class
  \KOMAoptions{parskip=half}}
\makeatother
\usepackage{xcolor}
\IfFileExists{xurl.sty}{\usepackage{xurl}}{} % add URL line breaks if available
\IfFileExists{bookmark.sty}{\usepackage{bookmark}}{\usepackage{hyperref}}
\hypersetup{
  pdftitle={Syllabus - Foundations of Biology 2 (BIOSCI 0160},
  pdfauthor={Nathan L Brouwer},
  hidelinks,
  pdfcreator={LaTeX via pandoc}}
\urlstyle{same} % disable monospaced font for URLs
\usepackage{longtable,booktabs,array}
\usepackage{calc} % for calculating minipage widths
% Correct order of tables after \paragraph or \subparagraph
\usepackage{etoolbox}
\makeatletter
\patchcmd\longtable{\par}{\if@noskipsec\mbox{}\fi\par}{}{}
\makeatother
% Allow footnotes in longtable head/foot
\IfFileExists{footnotehyper.sty}{\usepackage{footnotehyper}}{\usepackage{footnote}}
\makesavenoteenv{longtable}
\usepackage{graphicx}
\makeatletter
\def\maxwidth{\ifdim\Gin@nat@width>\linewidth\linewidth\else\Gin@nat@width\fi}
\def\maxheight{\ifdim\Gin@nat@height>\textheight\textheight\else\Gin@nat@height\fi}
\makeatother
% Scale images if necessary, so that they will not overflow the page
% margins by default, and it is still possible to overwrite the defaults
% using explicit options in \includegraphics[width, height, ...]{}
\setkeys{Gin}{width=\maxwidth,height=\maxheight,keepaspectratio}
% Set default figure placement to htbp
\makeatletter
\def\fps@figure{htbp}
\makeatother
\setlength{\emergencystretch}{3em} % prevent overfull lines
\providecommand{\tightlist}{%
  \setlength{\itemsep}{0pt}\setlength{\parskip}{0pt}}
\setcounter{secnumdepth}{5}
\usepackage{booktabs}
\ifLuaTeX
  \usepackage{selnolig}  % disable illegal ligatures
\fi
\usepackage[]{natbib}
\bibliographystyle{apalike}

\title{Syllabus - Foundations of Biology 2 (BIOSCI 0160}
\author{Nathan L Brouwer}
\date{2022-01-10}

\begin{document}
\maketitle

{
\setcounter{tocdepth}{1}
\tableofcontents
}
\hypertarget{welcome}{%
\chapter{Welcome!}\label{welcome}}

This is the syllabus for Dr.~Brouwer's section of Foundations of Biology 2. The syllabus is organized into sections accessible via the menu on the left. The first few sections are those students most often need to be familiar with at the beginning of the semester. After that sections are organized approximately alphabetically by major topics. You can search for terms using the ``Search'' panel in the top-left corner.

Some policies related to exams and assignments may not be finalized until the first day of class. I also reserve the right to update and modify the schedule as needed. All students will be informed verbally and via email about any changes.

\hypertarget{course-units}{%
\chapter{Course Units}\label{course-units}}

The course is divided into \textbf{3 Units.} Each Unit will be split into 2 Parts.

There will be 1 exam for each of the 3 Units of the course.

The first 3 Units will be focused on

\begin{itemize}
\tightlist
\item
  molecular genetics
\item
  mutation
\item
  gene regulation
\item
  genomics
\item
  biotechnology / lab techniques
\end{itemize}

The last 3 Units will be focused on

\begin{itemize}
\tightlist
\item
  phylogenetics
\item
  evolution
\item
  ecology
\end{itemize}

Each Unit will have approximately 8 lectures.

\hypertarget{material-in-the-course-is-cumulative}{%
\chapter{Material in the course is cumulative!}\label{material-in-the-course-is-cumulative}}

Content within the course is cumulative, with each exam referring back to concepts from previous material!

This integration will occur within the lectures also so you shouldn't be surprised by the connections being made.

The three most common recurring themes will be:

\begin{enumerate}
\def\labelenumi{\arabic{enumi}.}
\tightlist
\item
  Data analysis methods (scatterplots, histograms etc.)
\item
  Mutations
\item
  Phylogenetic trees.
\end{enumerate}

Practice tests and problems sets will be available so you will be able to see how the content in different units relates.

\hypertarget{communication-me2u}{%
\chapter{Communication - Me to you}\label{communication-me2u}}

\textbf{Messages via Canvas will be used to regularly communicate key information for the course.}

After the first week, I will generally send out a single message via Canvas on Friday evening after the last recitation. This will provide information about the relevant readings for the upcoming lectures, assignments, reminders about exams, etc.

Additional messages on time-sensitive issues (e.g.~class cancellation due to bad weather) or reminders about upcoming exams will also be released.

Please set your Canvas setting so that all Canvas messages are forwarded to your Pitt email. If you use another email service as your primary email (eg GMail) please set your Pitt email to forward there.
For info on forwarding your Pitt email to your personal account follow this {[}link{]}(\url{https://bit.ly/2Riz7dx} : \url{https://bit.ly/2Riz7dx}

I recommend adjusting your Canvas settings to what works best for you regarding the frequency of messages. A daily digest of Canvas messages is usually the best so you aren't constantly pinged by messages throughout the day.

All elements of the course will be scheduled to be released on specific days. Check the course calendar for upcoming due dates.

\hypertarget{communication-u2me}{%
\chapter{Communication - You to me}\label{communication-u2me}}

I you can contact me via email at \textbf{\href{mailto:nlb24@pitt.edu}{\nolinkurl{nlb24@pitt.edu}}} or via Canvas.

\begin{quote}
Please put ``\textbf{Foundations 2:\ldots{}}'' as the subject, with an informative bit of information as the ``\ldots{}''. e.g.~``Foundations 2: Canvas assignment not allowing multiple attempts''. There's no need to provide your PeopleSoft number.
\end{quote}

I try to answer all emails received on \emph{weekdays} within 24-36 hrs. Emails received on the weekend will be answered Monday.

Please consult the syllabus before asking questions about course policies and the schedule, and refer to relevant information such as web links, pages or dates. Screengrabs are super helpful. If the entire answer to your question can be found in the syllabus I will likely respond by saying ``This is in the syllabus, Cheers, Dr.~B.''

Questions relevant to the whole class may be re-posted, with identifying details removed.

\hypertarget{course-structure}{%
\chapter{Course structure}\label{course-structure}}

The following sections outline key elements of the course.

Further details will be provided later in the syllabus.

\hypertarget{weekly-course-components}{%
\section{Weekly Course Components}\label{weekly-course-components}}

Foundations of Biology 2 is an in-person course. There are 4 main forms of delivering material and interaction:

\begin{itemize}
\tightlist
\item
  \textbf{Lectures}: In-person, Tuesdays \& Thursdays
\item
  \textbf{Recitation}: Fridays.
\item
  \textbf{Office Hours} with Dr.~Brouwer throughout the week.
\item
  \textbf{Undergraduate Teaching Assistant (UTA) office hours} throughout the week.
\end{itemize}

\hypertarget{assessments}{%
\section{Assessments}\label{assessments}}

The primary form of assessment in the course will be 3 ``mini-tests'', each covering \textasciitilde8 lectures of material
Approximately 80\% of your grade will be based on these tests.

Other primary forms of assessment and practice include

\begin{itemize}
\tightlist
\item
  \textbf{Assignments}
\item
  \textbf{Practice tests} taken during recitation
\item
  \textbf{Exam rematches} taken during recitation
\item
  Other recitation activities to build skill and comprehensions
\item
  Numerous short quizzes before and after lecture to build vocab and test comprehension
\end{itemize}

As detailed elsewhere in the syllabus a portion of each of these individuals components of your grade will be dropped.

Sometimes assignments or questions will be marked as being based partially or entirely scored for participation. These assignments ARE required - ``for participation'' does NOT mean optional.

Occasionally, fully optional assignments worth 0 points will be released - these will be clearly marked as worth 0 points and for practice only.

The syllabus will take you step-by-step through all the policies related to these elements of the course.

\hypertarget{officehours}{%
\chapter{Office Hours}\label{officehours}}

\hypertarget{dr.-brouwer-office-hours}{%
\section{Dr.~Brouwer office hours}\label{dr.-brouwer-office-hours}}

Office hours will be held each week starting the second week of class. See the Canvas homepage for times. (These are subject to modification once the semester begins)

\textbf{Office hours are NOT the day after an exam, or finals week.}

Office hours may be available via Zoom but will not be recorded.

\textbf{Some guidelines for office hours:}

\begin{itemize}
\tightlist
\item
  Don't be intimidated by office hours. One of my favorite parts of my job is talking with students one-on-one.
\item
  Be prepared for office hours. Come with specific questions related to slides, figures, etc. But if you just want to listen, that's totally cool
\item
  Office hours are for everyone. If multiple people show up I will alternate between them to answer their questions. If I feel I have answered your main questions I may ask that we table our discussion to cover other questions.
\item
  Note: During finals week I do not hold office hours and am not available for appointments.
\item
  During office hours, specific questions are better than vague ones, even if it's just ``I specifically don't understand Fig. 4 in this chapter.'' A question like ``I don't understand natural selection'' is harder for me to work with.
\item
  I'm also happy to talk about study strategies, classes you might want to take, research opportunities etc.
\end{itemize}

\hypertarget{uta-office-hours}{%
\section{UTA Office Hours}\label{uta-office-hours}}

Undergraduate Teaching Assistants (UTAs) will hold office hours throughout the week by in-person and via Zoom.

These will be finalized by the 2nd week of class and summarized online.

You can attend any UTA's office hours. UTAs will have access to keys to help you review assignments and recitation materials.

UTA's will read the tests but will not have access to test materials.

\hypertarget{success}{%
\chapter{How to succeed in Foundations of Biology 2}\label{success}}

\begin{enumerate}
\def\labelenumi{\arabic{enumi}.}
\tightlist
\item
  \textbf{Have confidence} -- you can learn the material!\\
\item
  \textbf{Read the textbook pages} that will be covered before class.
\item
  \textbf{Read slowly and thoroughly, but taking notes isn't necessary}. Familiarize yourself with the vocabulary terms and look up unfamiliar words.
\item
  \textbf{Examine all indicated figures and read the figure captions}. Analysis of the information in figures plays a HUGE part in assignments and tests!
\item
  \textbf{Take notes in class}, and review them at least briefly within a few days. I recommend taking notes by hand with pen and paper. If possible, print out the slides and fill in key information.
\item
  \textbf{Fill in any unclear sections in your notes} with information from the book, other suggested study resources, information from the discussion boards or office hours, etc.
\item
  \textbf{Study with a partner or small study group}, using Zoom as necessary.
\item
  \textbf{Regularly attend office hours} with me or the UTAs (or both!), even if it's just to listen to what other people are asking.
\item
  \textbf{Try not to study for long, uninterrupted periods of time without a significant break.} Cramming results in - at best - short term retention of material. Two half-hour or 45-minute study sessions are better than one marathon 1-2 hour session.
\item
  \textbf{Ask for help when necessary - before you fall behind}. Go to office hours as often as necessary. The UTAs and I are here to help!
\end{enumerate}

\hypertarget{academic-integrity}{%
\chapter{Academic integrity}\label{academic-integrity}}

Dr.~Brouwer, the Bio Sci department, and the University all take academic integrity very seriously. Cheating includes \textbf{any} form of plagiarism, including copying other students' work or using other resources without proper attribution.

If you are caught cheating on a graded assignment, you will receive a zero on the assignment.

If you are caught cheating on an exam, you will receive a zero on the exam, an F in the course, and an Academic Integrity Violation Report will be filed.

Materials made available to you in this class are my intellectual property. They are for your private use only. Posting or sharing my materials (lecture notes, slides, quizzes, homework, exams, recitation assignments, etc.) to any website or with any student not currently enrolled in this course, without my express written permission, is a violation of the academic integrity code. I have accounts on these sites and monitor them.

Below is the University's Policy on Academic Integrity:

\begin{quote}
``Students in this course are expected to comply with the University of Pittsburgh School of Arts \& Sciences Academic Integrity Code located at www.as.pitt.edu/faculty/policy/integrity.html. Any student suspected of failing to meet the student obligations of the code during the semester will be required to participate in the procedures for adjudication, initiated at the instructor level. This may include, but is not limited to, confiscation of the assignment of any individual suspected of violating the code. A minimum sanction of a zero score for the assignment will be imposed. Violation of the Academic Integrity Code requires the instructor to submit an Academic Integrity Violation Report to the Dean.''
\end{quote}

\hypertarget{acheive}{%
\chapter{Acheive}\label{acheive}}

If you are accessing the textbook online through the information sent to you by the bookstore or purchased a new copy of the textbook you will have access to the publisher's online materials, called Achieve. (Note that there is an older service called LaunchPad which we no longer use).

I do NOT assign materials from Achieve but will make them available for relevant chapters. You may find them useful, but I cannot guarantee their relevance to the course and will not answer questions related to them.

If you use Achieve I recommend talking to the UTAs about which questions are most similar to what will appear on the tests.

Achieve can be accessed from within Canvas using the ``Macmillan Learning'' link on the menu to the left.

\hypertarget{individual-appointments-with-dr.-b}{%
\chapter{Individual appointments with Dr.~B}\label{individual-appointments-with-dr.-b}}

If you are unable to make office hours or have a matter you wish to discuss in private you can schedule an \textasciitilde15 minutes appointment using Calendly. See the Canvas home page for the link.

If none of the available times work please email me to set up a time.

\hypertarget{buffer-points-on-exams}{%
\chapter{Buffer points on exams}\label{buffer-points-on-exams}}

Exams, the final, and exam rematches all have \textbf{buffer points}.

This is how this works: Let's say exams this semester are worth 40 points. Each exam will have 41 questions, each worth 1 points. The maximum score on an exam will be 40 points, so you have 41 chances to earn up to 40 points.

(I may sometimes refer to these as ``buffer questions''; no particular question, however, is designated as the buffer.)

\hypertarget{course-communication-university-email-policy}{%
\chapter{Course Communication: University email policy}\label{course-communication-university-email-policy}}

Each student is issued a University e-mail address (\href{mailto:username@pitt.edu}{\nolinkurl{username@pitt.edu}}). This e-mail address may be used by the University for official communication. Students are expected to read e-mail sent to this account on a regular basis. Failure to read and react to University communications in a timely manner does not absolve the student from knowing and complying with the content of the communications. The University provides an e-mail forwarding service that allows students to read their e-mail via other service providers. Students that choose to forward their e-mail from their pitt.edu address to another address do so at their own risk. If e-mail is lost as a result of forwarding, it does not absolve the student from responding to official communications sent to their University e-mail address. To forward e-mail sent to your University account, go to \url{http://accounts.pitt.edu}, log into your account, click on Edit Forwarding Addresses, and follow the instructions on the page. Be sure to log out of your account when you have finished. (For the full E-mail Communication Policy, go to bc.pitt.edu/policies/policy/09/09-10-01.html.)

\hypertarget{disability-resources-and-services-drs}{%
\chapter{Disability Resources and Services (DRS)}\label{disability-resources-and-services-drs}}

\textbf{Disabilities Resources \& Services:}
216 William Pitt Union
(412) 648-7890
(412) 383-7355 (TYY)
\href{https://www.diversity.pitt.edu/disability-access/disability-resources-and-services}{Website}: \url{https://www.diversity.pitt.edu/disability-access/disability-resources-and-services}

If you have a disability for which you are or may be requesting an accommodation, you are encouraged to contact both your instructor and Disabilities Resources and Services. DRS will verify your disability and determine reasonable accommodations.

\hypertarget{exam-policies}{%
\chapter{Exam Policies}\label{exam-policies}}

\hypertarget{number-of-exams}{%
\section{Number of exams}\label{number-of-exams}}

\begin{enumerate}
\def\labelenumi{\arabic{enumi}.}
\tightlist
\item
  \textbf{There will be 3 mini-tests during the semester.} 1 will be dropped as detailed later in the syllabus.
\item
  \textbf{There are approximately 8 lectures and 3 recitations per unit and mini-test.} (Some units may have 3 recitations)
\end{enumerate}

\hypertarget{exam-drops}{%
\section{Exam drops}\label{exam-drops}}

\begin{enumerate}
\def\labelenumi{\arabic{enumi}.}
\tightlist
\item
  Your lowest scores of the 3 unit exams will be automatically dropped when grades are calculated at the end of the semester. (Prior to that it will be factored into your grade.)
\item
  \textbf{The final exam CANNOT be dropped.}
\item
  If you are unable to take a exam for ANY reason it will be counted zero.
\item
  If you miss just 1 test, that score will be automatically dropped as your lowest score when grades are calculated.
\item
  Your score on the final can replace your 2nd lowest score, if the final is higher. See section of the syllabus on the final exam.
\end{enumerate}

If you miss 2 tests, 1 will be dropped and the 2nd can be replaced by your score on the final.

\hypertarget{there-are-no-makeup-exams}{%
\section{There are no makeup exams}\label{there-are-no-makeup-exams}}

\textbf{No makeup exams} will be given for any reason.

\hypertarget{exams-will-be-in-person-and-most-likely-on-canvas}{%
\section{Exams will be in-person and most likely on Canvas}\label{exams-will-be-in-person-and-most-likely-on-canvas}}

\begin{itemize}
\tightlist
\item
  All exams, including the final, will be in-person.\\
\item
  \textbf{Canvas-based exams:} If/when exams are taken using Canvas, it is your responsibility to assure that you have adequate power to your device and internet access.
\item
  If you lose power or internet access before completing the test and receive a sub-optimal score then this score will likely be the test score that gets dropped.
\end{itemize}

\hypertarget{exam-length-format}{%
\section{Exam length \& format}\label{exam-length-format}}

NOTE: the following policies are provisional and may be adjusted as needed to optimize test length and how to implement the ``buffer question'' policy in Canvas.

\begin{itemize}
\tightlist
\item
  Exam will be scored out of 40 points. However, there will be 41 points worth of questions on the test.
\item
  Each test therefore will contain 1 buffer points that you can miss without harming your grade.
\item
  The maximum score on an exam will be capped at 40/40 If you get all 41 questions correct your score will be 100\%.
\item
  No extra credit will be offered on tests; the maximum score on a test will be 100\%.
\end{itemize}

\hypertarget{test-question-formats}{%
\section{Test question formats}\label{test-question-formats}}

\begin{itemize}
\tightlist
\item
  Test questions will frequently be multiple choice. Keyword: ``frequently'' - there will other types of questions too.
\item
  Most tests will have more than one question requiring a numeric calculation. You will be provided scratch paper for the exam.
\item
  Other common forms of questions are fill in the blank, dropdowns, or matching. This includes multiple fill in the blanks, multiple dropdowns, etc.\\
\item
  There will be NO True/False questions on tests.
\item
  For fill in the blank questions spelling errors will result in 0 points. No partial credit will be given.
\end{itemize}

\hypertarget{other-exam-policies}{%
\section{Other exam policies}\label{other-exam-policies}}

\begin{itemize}
\tightlist
\item
  When a test occurs on a Thursday, lecture material on Tuesday \textbf{WILL} occur on the test unless otherwise stated.
\item
  \textbf{Practice exam:} The recitation the Friday before a test will be used to go over a practice exam.
\item
  \textbf{I don't answer any questions during the test.} This is to preserve a quiet, consistent testing environment for all students.
\end{itemize}

\hypertarget{exam-review-rematches-reviews-retirees-and-exam-inquiries}{%
\chapter{Exam review: ``Rematches'', reviews, retirees and exam inquiries}\label{exam-review-rematches-reviews-retirees-and-exam-inquiries}}

After an exam you will have several opportunities to review the content of the exam.

\hypertarget{exam-rematches}{%
\section{Exam rematches}\label{exam-rematches}}

The first recitation after an exam will be used as a time for an ``\textbf{Exam Rematch}.''

During \textbf{Exam Rematches}, you will work in small groups to work through the questions on the exam again. You can discuss answers as a group, though each person will submit their own separate answers.

Exam rematches are graded for correctness, but are separate assignments and do NOT add points back to your score. They are a tool for review of the material while it is fresh in your memory.

The \textbf{Exam Rematch} assignment may contain all questions or the test, or be based just on the most missed and hardest questions of the preceding test.

\hypertarget{exam-review}{%
\section{Exam review}\label{exam-review}}

After grades are released the most missed questions will be discussed in class. You can take notes on the questions, but the full text of the questions will not be released at this time.

\hypertarget{exam-release-and-key}{%
\section{Exam release and key}\label{exam-release-and-key}}

After all steps of the exam review process your individual exams will be released for you to review. If you have further question about the exam an Exam Inquiry process will be established. Additional details will be released after the first exam review is over.

\hypertarget{exam-inquiries}{%
\section{Exam Inquiries}\label{exam-inquiries}}

If you have an inquiry about a test question you can fill out an Exam Inquiry Form after the key is released.
You will have until the Friday of the week the key is released to complete the form.
I will read all inquiries to determine if any immediate issue needs to be resolved. Inquiries will then be processed as detailed for Grade Inquiry Forms.
Unless you submitted an Exam Inquiry Form I will not discuss specific test questions during office hours.

\hypertarget{final-exam-policies}{%
\chapter{Final Exam Policies}\label{final-exam-policies}}

The final exam will \ldots{}

\begin{enumerate}
\def\labelenumi{\arabic{enumi}.}
\tightlist
\item
  Take place during finals week \emph{on the day scheduled by the University}
\item
  Be \textbf{in-person}
\item
  Be \textbf{cumulative}
\item
  Follow the general format of the mini-tests, including having buffer points
\item
  Potentially replace your second lowest score
\end{enumerate}

\hypertarget{final-exam-format}{%
\section{Final exam format}\label{final-exam-format}}

The final exam is split into *3** smaller exams. Each of those 3 exams totals up to your final grade on the final.

\hypertarget{the-final-exam-can-improve-second-lowest-exam-score}{%
\section{The final exam can improve second-lowest exam score!}\label{the-final-exam-can-improve-second-lowest-exam-score}}

I will compare your performance on each unit of the final to your performance on the previous unit exams, and determine which of the unit exams is your second-lowest score (the lowest will be dropped).

If your score on the appropriate section of the final is higher than your second lowest score, your score on that part of the final will replace your second lowest score.

Example:

Say your scores on the unit exams are

\begin{enumerate}
\def\labelenumi{\arabic{enumi}.}
\tightlist
\item
  0\%, because you missed the exam because you Omicron
\item
  72\%
\item
  80\%
\end{enumerate}

The lowest score (0\%) will be dropped automatically by Canvas prior to taking the final. Your exam grade on going into the final will therefore be based on the second and third exams: (72+80)/2 = 76\%.

Then say your scores on the 3 sections of the final are

\begin{enumerate}
\def\labelenumi{\arabic{enumi}.}
\tightlist
\item
  85\%
\item
  75\%
\item
  80\%
\end{enumerate}

Your overall score on the final will be (85+75+80)/2 = 80\%.

Your \emph{second} lowest unit exam score was 72\% (unit 2). Your score on the second part of the final (75\%) is higher Congratulations! So your score the unit exams will improve to (75+80)/2 = 77.5.

The final exam is weighted the same as the unit exams, so your overall score for exams plus the final will be: (75 + 80 + 80)/3 = 78.3\%

Unit exams plus the final are worth 80\% of your grade. Let's say you got 95\% on all other assignments.

Your final score will be = (exams+final) x 0.80 + (everything else) x 0.20.\\
That is, (78.3 x 0.80) + (95 x 0.2) = 81.64\%, which would be a B-.

\textbf{Note: }I do not hold office hours during finals week.

Information about the final exam schedule for the University can be found here:

\url{https://www.registrar.pitt.edu/students/final-exams}

\hypertarget{tests---practice-test}{%
\chapter{Tests - Practice test}\label{tests---practice-test}}

The recitation prior to an exam will be used to take prepare for the exam. A pre-recitation take-home practice test will be given as an assignment, a short practice test will be taken and discussed during recitation, and a follow-up post-recitation practice test will be assigned.

All of this will be required assignments. These will be based on test questions from previous years and will be administered using Canvas.

\hypertarget{extra-credit-tldr-there-is-none}{%
\chapter{Extra credit (tl;dr: there is none)}\label{extra-credit-tldr-there-is-none}}

No extra credit will be offered in this course.

``Buffer points'' on the exams and exam rematches are NOT extra credit.

\hypertarget{faq}{%
\chapter{FAQ}\label{faq}}

Welcome to Foundation 2 with Dr.~Brouwer! This FAQ supplements the syllabus and answers frequently asked questions. Let me know ( \href{mailto:nlb24@pitt.edu}{\nolinkurl{nlb24@pitt.edu}} ) if you have any additional questions.

\hypertarget{frequently-asked-questions}{%
\section{Frequently asked questions}\label{frequently-asked-questions}}

\hypertarget{are-we-starting-online}{%
\subsection{``Are we starting online?''}\label{are-we-starting-online}}

Yes We will begin meeting online for the first 2.5 weeks for both lecture and recitation. A Zoom link will be sent prior to the first day of class and will also appear on the front page of the course website.

\hypertarget{what-textbook-will-we-use}{%
\subsection{``What textbook will we use?''}\label{what-textbook-will-we-use}}

Our book for Foundations 2 is the 12th edition of Life: The Science of Biology by Hillis et al.~(2020).

Everyone should have received an email recently from the bookstore discussing the Inclusive Access program using Redshelf.

If you click on the Macmillan Learning tab on the Canvas sidebar for this course and select ``E-Book'' you should be taken to the online version of the book.

You have been charted a small fee for online access; if you do not need it you can opt out. Check your email for instructions.

See the Readings - Assigned textbook materials for more information.

\hypertarget{will-we-use-achieve}{%
\subsection{``Will we use Achieve?''}\label{will-we-use-achieve}}

No, we will not use Achieve for gradeds work. It will be activated so you can use it for practice questions. See the Achieve section of the syllabus for details.

\hypertarget{will-we-use-launchpad}{%
\subsection{``Will we use LaunchPad''}\label{will-we-use-launchpad}}

No, we will not use LaunchPad.

\hypertarget{will-any-content-be-asynchronous}{%
\subsection{``Will any content be asynchronous?''}\label{will-any-content-be-asynchronous}}

No, no content will be asynchronous. This is an in-person course that is beginning online the first 2.5 weeks. After that it will be fully in person. Videos will be provided for review.

\hypertarget{what-if-i-get-sick-this-semester}{%
\subsection{``What if I get sick this semester?''}\label{what-if-i-get-sick-this-semester}}

In general, for my courses I use a ``drop your lowest scores'' policy where -- regardless of the reasons -- your lowest score on each major part of your (tests, assignments, etc) gets dropped. This should accommodate an extended period of illness. See the full grade breakdown for details

\hypertarget{what-if-you-get-sick-this-semester}{%
\subsection{``What if YOU get sick this semester?''}\label{what-if-you-get-sick-this-semester}}

All of my materials have been prepared far enough in advance that I will be able to keep the course running even if I get sick using videos. I'm fully vaccinated and boosted and not expecting any disruption to the course.

\hypertarget{what-topics-will-we-be-covering}{%
\subsection{``What topics will we be covering?''}\label{what-topics-will-we-be-covering}}

The course covers molecular biology, genomics, evolution, and ecology. We'll start out with Chapters 13, 14, 15, 16, and Appendix B over the first several weeks.

\hypertarget{im-not-arriving-until-after-the-1st-day-of-class---is-that-a-problem}{%
\subsection{``I'm not arriving until after the 1st day of class - is that a problem?''}\label{im-not-arriving-until-after-the-1st-day-of-class---is-that-a-problem}}

No problem at all. Everyone will be participating via zoom the first 2.5 weeks.

\hypertarget{i-joined-the-class-late-what-can-i-do}{%
\subsection{``I joined the class late, what can I do?''}\label{i-joined-the-class-late-what-can-i-do}}

All assignments associated with Unit 1 will remain open for an extended period of time to accommodate people who join the class late.

\hypertarget{not-so-frequently-asked-question}{%
\section{Not so Frequently Asked Question}\label{not-so-frequently-asked-question}}

\hypertarget{i-want-to-read-ahead-where-should-i-start}{%
\subsection{``I want to read ahead, where should I start''}\label{i-want-to-read-ahead-where-should-i-start}}

Below are the chapters and key sections covered in the first unit of the course. They are in the approximate order we'll cover them.

\textbf{Readings from textbook:}

\begin{itemize}
\tightlist
\item
  Appendix B: Making sense of data: A statistics Primer, Sections ``Step 3'' through ``Step 4.''
\item
  Chapter 20: Reconstructing and using phylogenies, Section 20.1
\item
  Chapter 13: DNA and its role in heredity - sections 13.2 to 13.5 (skip 13.1)
\item
  Chapter 14: From DNA to protein - section 14.2 to 14.6 (skip 14.1)
\item
  Chapter 15: Gene mutation and molecular medicine, section 15.1 (skip all other sections)
\end{itemize}

\textbf{Additional Readings:}

\begin{itemize}
\tightlist
\item
  How Histograms Work: \url{https://bit.ly/biohistograms}
\item
  How to Read \& Use a Boxplot: \url{https://bit.ly/bioboxplots}
\end{itemize}

\hypertarget{what-do-i-need-to-know-from-previous-biology-classes}{%
\subsection{``What do I need to know from previous biology classes?''}\label{what-do-i-need-to-know-from-previous-biology-classes}}

See the page ``Foundations 1 - things to know'' which lists chapters and focal ideas covered in Pitt's first semester of biology (Foundations 1) that are most relevant to Foundations 2.

Depending on your background you may wish to glance over them.

\hypertarget{general-education-requirement}{%
\chapter{General Education Requirement}\label{general-education-requirement}}

This course fulfills 1 Dietrich School of Arts and Sciences Natural Science General Education Requirement (GER) as described for the GERs starting Fall 2018 (term 2191). That GER reads as follows:

\begin{quote}
``Three Courses in the Natural Sciences: These will be courses that introduce students to scientific principles and concepts rather than offering a simple codification of facts in a discipline or a history of a discipline. The courses may be interdisciplinary, and no more than two courses may have the same primary departmental sponsor.''
\end{quote}

See \href{https://asundergrad.pitt.edu/academic-experience/general-education-requirements}{asundergrad.pitt.edu/academic-experience/general-education-requirements} for more details.

\hypertarget{goals-of-foundations-2}{%
\chapter{Goals of Foundations 2}\label{goals-of-foundations-2}}

The goals of Foundations II is to expand your understanding of key biological concepts and build skills essential to your success as a practitioner and consumer of science.

\textbf{There is a strong emphasis in this course on application} - there is MUCH more to this class than memorizing vocab!

\hypertarget{concepts-from-nucleic-acids-to-the-biosphere}{%
\section{Concepts: From Nucleic acids to the Biosphere}\label{concepts-from-nucleic-acids-to-the-biosphere}}

Key biology concepts covered this semester will be

\begin{enumerate}
\def\labelenumi{\arabic{enumi}.}
\tightlist
\item
  DNA structure \& function
\item
  Transcription \& translation
\item
  Genomics \& DNA cloning
\item
  \href{https://en.wikipedia.org/wiki/Phylogenetic_tree}{Phylogenetic trees}
\item
  Evolution \& speciation
\item
  Ecology
\end{enumerate}

Biology is the study of life in all its forms and at all scales, from the interactions of biological macromolecules like nucleic acids to the functioning of the entire biosphere. Foundations II will provide you with a grounding in the major biological concepts related to the molecular basis of genetics, molecular biology, the evolution of life, and ecological interactions.

\hypertarget{molecular-genetics}{%
\subsection{Molecular genetics}\label{molecular-genetics}}

We will begin the semester with an introduction to the molecular basis of genetics and heredity, focusing on how the information in DNA is stored, copied, transmitted, and used by organisms through the processes of transcription and translation.

\hypertarget{gene-regulation-genomics}{%
\subsection{Gene regulation \& genomics}\label{gene-regulation-genomics}}

During the first units we will think mostly in terms of single genes. Later we will widen our scope to include multiple genes in order to examine gene regulation, genomics and \href{https://en.wikipedia.org/wiki/Epigenetics}{epigenetics} We'll also discuss how molecular biologists conduct experiments and the tools they use to manipulate genes and proteins.

\hypertarget{population-biology-natural-selection-speciation-phylogenetics-ecology}{%
\subsection{Population biology: natural selection, speciation, phylogenetics \& ecology}\label{population-biology-natural-selection-speciation-phylogenetics-ecology}}

Throughout the course we will emphasize the evolutionary background of the topics we discuss. During the second half of the course we will specifically focus on the biology of populations and the dynamics of natural selection, the origin of species (speciation), the evolutionary relationships among species (\href{https://en.wikipedia.org/wiki/Phylogenetics}{phylogenetics}), and the process of population growth and regulation which are key to understanding evolution (population ecology).

\hypertarget{species-interactions-community-ecology}{%
\subsection{Species interactions \& Community ecology}\label{species-interactions-community-ecology}}

We will finish the semester learning about species interactions, such as predators versus their prey, and the structure of communities made up of multiple interacting species.

\hypertarget{skills-doing-communicating-science}{%
\section{Skills: Doing \& Communicating Science}\label{skills-doing-communicating-science}}

Foundations 2 will develop your skills as a critical consumer and practitioner of science. By the end of the course you should not only understand the conceptual material presented in the book but also be able to critically approach new scientific material such as scientific talks and journal articles. We will practice assessing scientific results, understanding figures, and presenting and analyzing data. Throughout the course we will therefore discuss

\begin{itemize}
\tightlist
\item
  How science is conducted \& communicated
\item
  Application of the scientific method
\item
  How experiments \& observational studies are conducted
\item
  Understanding scientific figures
\item
  Data analysis \& Statistics
\item
  Scientific inference
\end{itemize}

\hypertarget{grades---grading-scale}{%
\chapter{Grades - grading scale}\label{grades---grading-scale}}

Raw percentage grades will be converted to letter grades at the end of the semester using the scale shown below. This will only be done after the final and the implementation of all dropped grades.

Note: Students planning to major in Biological Sciences must pass this course with a \textbf{C} (not C- !) or better.

\textbf{Rounding}: Rounding is not done until final grades are computed and is done by computer to 1 decimal place. Final letter grades are assigned after rounding and is done automatically by a computer including the decimal value. For example, a score of 91.99\% rounds to 92.0\% and is an A, but a score of 91.94\% rounds to 91.9\% and is an A-.

\begin{longtable}[]{@{}
  >{\centering\arraybackslash}p{(\columnwidth - 6\tabcolsep) * \real{0.17}}
  >{\centering\arraybackslash}p{(\columnwidth - 6\tabcolsep) * \real{0.26}}
  >{\centering\arraybackslash}p{(\columnwidth - 6\tabcolsep) * \real{0.11}}
  >{\centering\arraybackslash}p{(\columnwidth - 6\tabcolsep) * \real{0.11}}@{}}
\toprule
\begin{minipage}[b]{\linewidth}\centering
~
\end{minipage} & \begin{minipage}[b]{\linewidth}\centering
Final Percentage
\end{minipage} & \begin{minipage}[b]{\linewidth}\centering
Grade
\end{minipage} & \begin{minipage}[b]{\linewidth}\centering
GPA
\end{minipage} \\
\midrule
\endhead
\textbf{row01} & 98.0--100\% & A+ & 4 \\
\textbf{row02} & 92.0--97.9 & A & 4 \\
\textbf{row03} & 90.0--91.9 & A- & 3.75 \\
\textbf{row04} & 88.0--89.9 & B+ & 3.25 \\
\textbf{row05} & 82.0--87.9 & B & 3 \\
\textbf{row06} & 80.0--81.9 & B- & 2.75 \\
\textbf{row07} & 78.0--79.9 & C+ & 2.25 \\
\textbf{row08} & 72.0--77.9 & C & 2 \\
\textbf{row09} & 70.0--71.9 & C- & 1.75 \\
\textbf{row10} & 68.0--69.9 & D+ & 1.25 \\
\textbf{row11} & 62.0--67.9 & D & 1 \\
\textbf{row12} & 60.0--61.9 & D- & 0.75 \\
\textbf{row13} & 59.0 and below & F & 0 \\
\bottomrule
\end{longtable}

\hypertarget{lecture-slides}{%
\chapter{Lecture slides}\label{lecture-slides}}

Lecture slides will usually be provided as Google Slides for each lecture, though at times it will be necessary for you to just take notes without access to the slides.

Google slides can be converted to PDF and PowerPoint.

The \emph{target} time for releasing lectures slides will be 5:00 PM on Mondays.

Lecture slides are released as a convenience, and I \textbf{cannot} guarantee that all or - or even any - slides will available prior to class.

You should always come to class prepared to take notes on material not available on shared slides.

Moreover, I often emphasize drawing diagrams by hand and will ask you to turn in your versions of figures you've drawn for participation credit.

\hypertarget{mental-health-wellness}{%
\chapter{Mental Health \& Wellness}\label{mental-health-wellness}}

School is hard - please take care of yourself as best you can given the many demands on your time. Diminished mental health, including significant stress, mood changes, excessive anxiety, or problems with sleeping can interfere with your academic performance. You have a support network here at Pitt to help you through challenging times. Acknowledging that you need help, and getting that help, is smart and courageous.

\begin{itemize}
\tightlist
\item
  If you are in an EMERGENCY situation, call 911 or Pitt Police at 412-624-2121.
\item
  If your symptoms are due to FINANCIAL strain, please visit pitt.libguides.com/assistanceresources to see all available University resources.
\item
  If your symptoms are due to strained RELATIONSHIPS, families, or personal crises, please visit the University Counseling Center at www.studentaffairs.pitt.edu/cc/ for free confidential services.
\item
  If your symptoms are strictly related to your COURSE WORK and performance in this course, please contact me.
\end{itemize}

\textbf{Important numbers:}

\begin{itemize}
\tightlist
\item
  University Counseling Center: 412-648-7930
\item
  Sexual Assault Response: 412-648-7856
\item
  RE:SOLVE crisis network: 888-796-8226
\item
  Pitt Police: 412-624-2121
\item
  \href{https://www.studentaffairs.pitt.edu/pittserves/the-pitt-pantry/}{Pitt Pantry (food bank)}: \url{https://www.studentaffairs.pitt.edu/pittserves/the-pitt-pantry/}
\end{itemize}

\hypertarget{point-allocation-weighting}{%
\chapter{Point allocation \& Weighting}\label{point-allocation-weighting}}

The course contains 3 units, and each unit has a test.

Your lowest scores will be automatically dropped as detailed elsewhere in the syllabus. (This will not be dropped until final grades are calculated do your grade reported by Canvas will be an under estimate with regards to tests!)

The exams and the final contribute \textasciitilde80\% of your final grade. Each will therefore be worth about 26.7\% of your grade. (\textasciitilde26.7*3 = \textasciitilde80\%)

Recitation homework, test revisions, other homework assignments, and quizzes will contribute \textasciitilde20\%.

The full point allocation is shown below. This is subject to change as necessary if, for example, the number of assignments changes.

A spreadsheet version of this table is available \href{https://docs.google.com/spreadsheets/d/1gB98V5qCye0tm26qK9Lho6NRVmTnoJ4COymKpLRL4Uc/edit\#gid=2147169333}{here}.

\url{https://docs.google.com/spreadsheets/d/1gB98V5qCye0tm26qK9Lho6NRVmTnoJ4COymKpLRL4Uc/edit\#gid=2147169333}

\hypertarget{points}{%
\chapter{Point values are relative}\label{points}}

As discussed in the Point Breakdown section, there are different categories of assignments, such as Exams, Pre-recitations, etc.

Each of these categories carries a different weight in the grading, with Exams and the final being the largest.

Therefore, the value of a ``point'' on an assignment will depend on which type of assignment it is part of. For example, a 1-point question on an Exam is worth MUCH more than a 1-point question on a recitation assignment.

\hypertarget{additional-readings}{%
\chapter{Readings: Additional readings}\label{additional-readings}}

\textbf{Many weeks there will be one or more \emph{required} short to medium-length additional reading to supplement the book. }

Links and/or PDFs for these materials will be provided. Many will be posted on \href{https://brouwern.github.io/fbes/}{Foundations of Biology and Environmental Science: An Open-Access Encyclopedia} \url{https://brouwern.github.io/fbes/} .

These readings will associated with the upcoming recitation or lecture, and some will be from the primary scientific literature.

\begin{quote}
I have carefully selected and often edited these additional readings and I highly recommend that you read them AND refer back to them when reviewing for tests.
\end{quote}

Optional \textbf{review readings} may also be posted prior to tests. These will be clearly marked as ``optional.''

\hypertarget{textbook}{%
\chapter{Readings: Assigned textbook materials}\label{textbook}}

This course will be structured around focal chapters and/or sections in Life: The Science of Biology, 12th edition by Hillis et al (2020).

\hypertarget{readme-assigned-readings}{%
\section{README: Assigned readings}\label{readme-assigned-readings}}

For a given lecture I will specify focal sections of the book, which importantly \textbf{may come from more than one chapter.} I will be very specific about what to read.

There are many sections, figures, and tables I will tell you to skip (hooray!).

After the first week, information on readings will be announced on Friday.

\begin{quote}
\textbf{I highly recommend reading the assigned sections \emph{before} class; however, I DO NOT recommend taking extensive notes or outlining chapters.} This applies to the first time you read it or as a study tactic. Read the assigned sections to prime your brain for lecture, then refer back to it when reviewing your notes.
\end{quote}

To facilitate review, most lectures will indicate which sections of the book are most related.

\hypertarget{digital-access-to-hillis-et-al-2020}{%
\section{Digital access to Hillis et al (2020)}\label{digital-access-to-hillis-et-al-2020}}

You will receive information about \emph{online} access to the 12th edition of the textbook. You can opt out of online access if you

\hypertarget{acheive-1}{%
\subsection{Acheive}\label{acheive-1}}

The publisher of the textbook has an interactive website of review materials called Achieve. I will make this materials available but nothing from this site will be assigned.

\hypertarget{hardcopy-of-book}{%
\section{Hardcopy of book}\label{hardcopy-of-book}}

You can obtain a physical copy of the book if you want. The 12th edition of the book -- which is green and has a bee on the cover -- is very similar to the 11th edition of the book by Sadava et al.~(2016), which is orange has a flock of birds on the cover.

I will indicate any important differences between the 12th and 11th edition of the books.

\hypertarget{rounding-numeric}{%
\chapter{Rounding numeric}\label{rounding-numeric}}

\hypertarget{entering-numeric-questions-on-tests-and-assignments}{%
\section{Entering numeric questions on tests (and assignments)}\label{entering-numeric-questions-on-tests-and-assignments}}

Test questions and assignments will frequently require numeric answers.

The keys for these questions will have a buffer to account for reasonable variation in rounding.

For example, if the answer is 50\%, grading will be set so that answers from 49\% to 51\% will be accepted.

\hypertarget{rounding-like-a-biologist}{%
\section{Rounding like a biologist}\label{rounding-like-a-biologist}}

Because chemistry and physics are super precise sciences, chemists and physics profs get really hung up on rounding; biologists generally don't unless its really necessary.

For rounding and reporting answers I recommend these rule of thumb:

\begin{enumerate}
\def\labelenumi{\arabic{enumi}.}
\tightlist
\item
  Your final answer doesn't usually need to be than 1 or 2 digits more precise than your data. So if I give you frequencies of 0.4 and 0.6 for a Hardy-Weinberg problem, your final answer can be rounded to the hundredths or thousandths (e.g.~0.411, 0.62).
\item
  If the final answer is \textgreater0, round to 2 digits past the decimal place, e.g.~round 1.519 to 1.52, and leave 1.49 as 1.49 (don't up round to 1.5).
\item
  If the final answer is \textless0, round to 2 digits past the zeros, e.g., round 0.00148 to 0.0015 and leave 0.0014 as 0.0014 (don't round to 0.001).
\item
  To prevent rounding errors retain 2 more digits than needed while doing your work.
\end{enumerate}

These aren't hard and fast rules; luckily I don't think I've ever had a problem on a numeric question that ultimately was the result of variation in approaches to rounding.

When making the key to a test I will usually try to work out the math with more than one approach to rounding to account for potential variation.

For example, I'll first do the calculations in a spreadsheet and do the rounding at the end.

I'll then redo the math rounding any intermediary steps. I'll then set a buffer that accommodates reasonable variation in the answer.

I also go back and specifically re-check numeric questions after the test.

\hypertarget{semester-schedule}{%
\chapter{Semester schedule}\label{semester-schedule}}

The semester schedule can be viewed on Google Drive \href{https://docs.google.com/spreadsheets/d/1gB98V5qCye0tm26qK9Lho6NRVmTnoJ4COymKpLRL4Uc/edit?usp=sharing}{here}:
\url{https://docs.google.com/spreadsheets/d/1gB98V5qCye0tm26qK9Lho6NRVmTnoJ4COymKpLRL4Uc/edit?usp=sharing}

\hypertarget{weekly-schedule}{%
\chapter{Weekly schedule}\label{weekly-schedule}}

The weekly schedule is available as a spreadsheet \href{https://docs.google.com/spreadsheets/d/1gB98V5qCye0tm26qK9Lho6NRVmTnoJ4COymKpLRL4Uc/edit?usp=sharing}{here}:
\url{https://docs.google.com/spreadsheets/d/1gB98V5qCye0tm26qK9Lho6NRVmTnoJ4COymKpLRL4Uc/edit?usp=sharing}

\hypertarget{study-guides-for-exams}{%
\chapter{Study guides for exams}\label{study-guides-for-exams}}

When preparing for the exams the best guides for studying are:

\begin{enumerate}
\def\labelenumi{\arabic{enumi}.}
\tightlist
\item
  Lecture slides
\item
  Your notes
\item
  Recitation homework assignments
\item
  Other homework assignments and quizzes
\item
  Vocab lists associated with slides, readings, and assignments
\end{enumerate}

\textbf{Exam studies guides} in the more traditional sense will be made available before each test -- \textbf{BUT these study guides are in no way complete, comprehensive, or representative}. They are most useful to look over AFTER you have thoroughly reviewed all your other class materials.

If you only look at the study guides you will likely miss a lot of important material.

\hypertarget{tophat}{%
\chapter{TopHat}\label{tophat}}

\protect\hyperlink{tophat}{TopHat}www.tophat.com(www.tophat.com) will be used for in-class activities.

The join code is located on the Canvas home page.

\hypertarget{updates-to-schedule-schedule-syllabus}{%
\chapter{Updates to schedule schedule \& syllabus}\label{updates-to-schedule-schedule-syllabus}}

I reserve the right to update the syllabus, schedule, point allocation and all other components of the course as necessary.

If changes occur after the first day of class, they will be clearly communicated in class and via email/Canvas message, and a revised syllabus and schedule distributed.

After reading the syllabus and completing the ``Syllabus treasure hunt'' assignment (which will be posted on the first day of class) feel free to contact me with any questions about course policies.

  \bibliography{book.bib,packages.bib}

\end{document}
